A student posed the question: Suppose the particle was at rest (zero momentum in the reference frame of observation), what happens to the De Broglie wavelength of the particle?

To answer this, we use the following observation of quantum mechanics: the particle is never at rest, it always has some non-zero velocity. It's kinetic energy can never be zero.

Can also be answered from the thermodynamic point of view: A particle cannot reach a temperature of absolute zero in a finite number of steps (3rd Law).

\begin{align*}
  E_{kin} = \frac{mv^2}{2} = \frac{p^2}{2m}~~~~\text{Particle}
\end{align*}

\begin{align*}
  E_{kin} = \frac{1}{2}
\end{align*}

For a photon, we assign a momentum:

\begin{align*}
  p = \frac{h}{\lambda} = \frac{h\nu}{c} = \frac{h}{cT} = \frac{\omega h}{2\pi c} = \frac{\bar{h}}{k}
\end{align*}

There are a number of relationships that link parameters of distance and time for a propagating wave. Remember them.

The energy of a photon, i.e it's kinetic energy is thus:

\begin{align*}
  E = n\bar{h}\nu
\end{align*}


